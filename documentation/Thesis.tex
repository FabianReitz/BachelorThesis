% ==========================================================
%
% Dies ist die LaTeX-Datei für das dritte Praxisprojekt von:
%
%                		Fabian Reitz
%
%				       mit dem Titel:
%
%  					 "Praxisprojekt 3"
%					
%			 in Kooperation mit stadt.werk GmbH
%
% ==========================================================

% ----------------------------
%
% Konfiguration des Dokumentes
%
% ----------------------------

% Dokumenten-Klasse und Format auf A4 festlegen:
\documentclass[a4paper]{scrartcl}

% Einen Zähler für Paragraphen benutzen:
\addtocounter{tocdepth}{1}

% Einrückung bei Absatzbeginn verhindern:
\setlength{\parindent}{0pt}

% ------------------
%
% Benötigte Packages
%
% ------------------

% Europäisches Buchstaben-Encoding verwenden:
\usepackage[T1]{fontenc}
\usepackage{csquotes}

% Deutsche Rechtschreibung nach aktueller Reform verwenden:
\usepackage[ngerman]{babel}

% Fontspec verwenden:
\usepackage{fontspec}

% Zeilenabstand verwenden:
\usepackage{setspace}

% Abschnitte im Dokument verlinken: 
\usepackage{hyperref}
\hypersetup{
	pdfencoding=auto,
	pdftitle={TITEL},
	pdfsubject={THEMA},
	pdfauthor={Fabian Reitz},
	pdfkeywords={},
   	hidelinks
}

% Befehle zum Festlegen der aktuellen Uhrzeit verwenden:
\usepackage{scrtime}

% Einbinden von Grafiken:
\usepackage{graphicx}
\usepackage{float}
\usepackage{sidecap}

% Setzen der Seitenabstände
\usepackage[a4paper, left=4cm, right=4cm, top=3cm, bottom=1.5cm]{geometry}

% Ein Abkürzungsverzeichnis benutzen:
\usepackage{acronym}

% Den Header des Dokumentes bearbeiten:
\usepackage{fancyhdr}

% Inhaltsverzeichnis und Abbildungsverzeichnis in 
% das Inhaltsverzeichnis einbinden:
\usepackage[notbib]{tocbibind}

% Setzen von Punkten als Trenner in das Inhaltsverzeichnis:
\usepackage[titles]{tocloft}
\renewcommand{\cftsecleader}{\cftdotfill{\cftdotsep}}

% TOC verbessern:
\usepackage{tocloft}

\setlength{\cftbeforesecskip}{3pt}

\usepackage{makecell}

\usepackage{titlesec}

\setcounter{secnumdepth}{4}

\titleformat{\paragraph}
{\normalfont\normalsize\bfseries}{\theparagraph}{1em}{}
\titlespacing*{\paragraph}
{0pt}{3.25ex plus 1ex minus .2ex}{1.5ex plus .2ex}

\usepackage{amsmath}

\usepackage{chngpage}

\usepackage{float}

% Speziell für die Thesis:
% ------------------------

% Biblatex zum Erstellen von Quellenangaben:
\usepackage{biblatex}

\addbibresource{bibliography.bib}


% --------------------------------
%
% Einstellungen für die Schriftart
%
% --------------------------------

% Schriftart auf "Times New Roman" festlegen:
\setmainfont[
	Path = ./_fonts/,
    BoldFont = Times-New-Roman-Bold.ttf,
    ItalicFont = Times-New-Roman-Italic.ttf
]{Times-New-Roman.ttf}

% Header bearbeiten:
\pagestyle{fancy}
\fancyhf{}
\fancyhead[C]{- \thepage\ -}
\renewcommand{\headrulewidth}{0pt}

\linespread{1.5}



% ---------------------
%
% Anfang des Dokumentes
%
% ---------------------

\begin{document}


% ---------------------
% Deckblatt definieren:
% ---------------------



\titlehead{\centering\includegraphics[scale=0.3]{_assets/logo_DHSH}}
\subject{Praxisprojekt 3 - Informatik \\}
\title{Digitalisierung eines analogen Prozesses unter Anwendung moderner UI- und UX-Designmethoden}
\subtitle{- am Beispiel einer webbasierten Anwendung zur Erstellung von Rechnungen}
\author{\date{}}
\vspace{2cm}
\publishers{
	\begin{tabbing}
		Betreuender Dozent: tab \= Mitte \= Rechts \= \kill
		
		Abgegeben von: 	\> Fabian Reitz \\
		E-Mail: 		\> fabian.reitz@stud.dhsh.de \\
		Studiengruppe:	\> WINF 119 B \\ \\
		Gutachter:		\> Herr Prof. Dr. Alexander Paar \\ \\
		Abgabetermin:	\> 13.01.2022 \\
		Versuch:		\> Erstversuch \\
	\end{tabbing}
}
\maketitle

\thispagestyle{empty}

% -------------------
% Inhaltsverzeichnis:
% -------------------

\newpage

% Alle folgenden Seiten in römischen Zahlen zählen:
\pagenumbering{Roman}

% Beginn der Paginierung bei zwei:
\setcounter{page}{2}

% Inhaltsverzeichnis zeigen:
\tableofcontents


% ----------------------
% Abkürzungsverzeichnis:
% ----------------------

% Neue Seite beginnen:
\newpage

\section*{Abkürzungsverzeichnis}

\addcontentsline{toc}{section}{Abkürzungsverzeichnis}

% Abkürzungsverzeichnis beginnen:
\begin{tabbing}
	----------------------- \= Erklärung \kill
	e.V. \> eingetragener Verein \\
	GmbH \> Gemeinschaft mit beschränkter Haftung \\
	GUI \> Graphical User Interface \\
	\> deutsch: Grafische Nutzungsoberfläche \\
	PDF \> Portable Document Format, ein Dateistandard zur Handhabung von \\ \> Dokumenten \\
	UI (-Design) \> User Interface (Design)  \\
	\> deutsch: Nutzungsoberfläche oder Erstellung der Nutzungsoberfläche \\
	UX (-Design) \> User Experience (Design) \\
	\> deutsch: Nutzungserfahrung oder Erstellung der Nutzungserfahrung \\ 
\end{tabbing}


% ----------------------
% Abbildungsverzeichnis:
% ----------------------
\newpage

\listoffigures


% --------------------
% Tabellenverzeichnis:
% --------------------
\newpage

\listoftables

% -----------
% Einleitung:
% -----------

% Neue Seite erstellen:
\newpage

% Paginierung mit eins beginnen:
\setcounter{page}{1}

% Alle folgenden Seiten in arabischen Zahlen zählen:
\pagenumbering{arabic}

% Neue Section: Einleitung
\section{Einleitung}

% ---------------------
% Literaturverzeichnis:
% ---------------------



% Neue Seite beginnen:
\newpage

% Seitenzahl als römische Zahl angeben:
\pagenumbering{Roman}

% Beginn der Zählung bei 5:
\setcounter{page}{6}

% Literaturverzeichnis zu TOC hinzufügen:
\addcontentsline{toc}{section}{Liteaturverzeichnis}

% Literaturverzeichnis:
\section*{Literaturverzeichnis}

% Einfacher Zeilenabstand:
\singlespacing

% --------------------------
% Eidesstattliche Erklärung:
% --------------------------

% Neue Section:
\section*{Eidesstattliche Erklärung}

\addcontentsline{toc}{section}{Eidesstattliche Erklärung}

% 1,5-facher Zeilenabstand:
\onehalfspacing

Ich erkläre an Eides Statt, dass ich meine Hausarbeit „Digitalisierung eines analogen Prozesses unter Anwendung moderner UI- und UX-Designmethoden - am Beispiel einer webbasierten Anwendung zur Erstellung von Rechnungen“ selbstständig und ohne fremde Hilfe angefertigt habe und dass ich alle von anderen Autoren wörtlich übernommenen Stellen wie auch die sich an Gedankengänge anderer Autoren eng anlehnenden Ausführungen meiner Arbeit besonders gekennzeichnet und die Quelle nach den mir von der Dualen Hochschule Schleswig-Holstein angegebenen Richtlinien zitiert habe. \\ \\

Kiel, den 13.01.2022 \\ \\ 

% Tabbing-Umgebung für Unterschriften:
\begin{tabbing}
	\_\_\_\_\_\_\_\_\_\_\_\_\_\_\_\_\_\_\_\_\_\_\_\_\_ \\
	Fabian Reitz
\end{tabbing}

% -------
% Anhang:
% -------

% Neue Seite beginnen:
\newpage

\appendix
% Neue Section:
\section{Prozess: Erstellung einer Rechnung}


\end{document}



















